\documentclass[9pt, addpoints]{exam}
\usepackage[english]{babel}
\usepackage[utf8x]{inputenc}
\usepackage{graphicx,lastpage}
\usepackage{hyperref}
\usepackage{amsmath}
\usepackage{amsthm}
\usepackage{amsfonts}
\usepackage{amssymb}
\usepackage{scrextend}
\usepackage{mathrsfs}
\usepackage{hhline}
\usepackage{booktabs} % book-quality tables
\usepackage{units}    % non-stacked fractions and better unit spacing
\usepackage{multicol} % multiple column layout facilities
\usepackage{lipsum}   % filler text
\usepackage{varwidth} % centering for itemize
\usepackage{listings}
\usepackage[linewidth=1pt]{mdframed}

\renewcommand{\qedsymbol}{$\blacksquare$}

\qformat{\thequestion\dotfill \emph{\totalpoints\ points}}
\pagestyle{headandfoot}
\header{T-409-TSAM}{Assignment 1}{\thepage/\numpages}
\runningheadrule
\firstpagefooter{}{}{}
\runningfooter{}{Page \thepage\ of \numpages}{}

\graphicspath{{../}{Figures/}}
\title{Assignment 1}

\begin{document}
\noindent
\begin{minipage}[l]{.11\textwidth}%
\noindent
    \includegraphics[width=\textwidth]{HR}
\end{minipage}%
%\hfill
\begin{minipage}[r]{.6\textwidth}%
\begin{center}
    {\large\bfseries Department of Computer Science \par
    \large Computer Networks \\[2pt]
    \large Due: Sunday 25 August (23.59)
    }
\end{center}
\end{minipage}%
\fbox{\begin{minipage}[l]{.4\textwidth}%
\noindent
    {\bfseries Your name:}\\[2pt]
TA Name:    \\
Time Taken: \\
{\footnotesize Estimated Time: {15 hours}}
\end{minipage}}%

\large     
\vspace{2cm}
\begin{center}
    \begin{minipage}{40em}
        \begin{center}
            This is an individual assignment and should be submitted as a pdf, with accompanying code, using Canvas.  
        \end{center}
        
        \vspace{6pt}
    For those who like to dabble in the dark arts, the latex version 
    is also available. You may submit in any legible form you wish, but please
    use tar to bundle files.
    
        \vspace{6pt}
    Practical programming exercises may be done on your local laptop, or
    using your account on skel.ru.is. Part of this assigment is getting your
    programming environment setup for this course. We strongly recommend that you
    create a suitable environment on your laptop or other machine which
    you can use to run client and monitoring software such as tcpdump.
    If you have any issues at all in getting setup, please contact 
    us \textbf{immediately}.

        \vspace{6pt}
    Marks are awarded for question difficulty. While there is 
    typically a relationship between difficulty and length of answer,
    it may not be a strong one. Always justify your answer if necessary,
    especially with somewhat open ended design questions.

    \vspace{6pt}
    Optional: Please include a rough estimate of how long it took you do 
    the assignment so that we can calibrate the work being assigned 
    for the course. (The estimated time is provided purely as a guideline.)
    \par
    \vspace{12pt}
    \end{minipage}
\end{center}

\vspace{4cm}
\begin{center}
    \gradetable[h]
\end{center}
\newpage
\section*{Introduction}

\begin{table}
\begin{tabular}{|ll|ll|}
    \hline
    \multicolumn{4}{|c|}{} \\
    \multicolumn{4}{|c|}{\textbf{Useful Network Commands (Linux)}}\\[6pt]
    \hline
    man    & Online manual for command              & man -k         & keyword search on manual            \\
    whois             & Domain registry information & htop           & Enhanced top, show processes        \\
    iftop             & Show top traffic/network    & iptraf         & IP traffic monitor                  \\
    route             & show local routing          & arp -v         & Show address resolution cache(root) \\
    nslookup          & interactively query DNS     & set type=A     & Specify A records                   \\
                      &                             & server=        & Specify server to query             \\
    dig               & DNS record lookup           & dig $<domain>$ & Get full record for $<domain>$      \\
    ncat $<ip> <port>$& Connect to remote system    & nmap           & Scan remote host                    \\
    netstat           & Show network statistics     & netstat -antup & Process path info                   \\
    netstat -t        & Show only tcp connections   & netstat -u     & show only udp connections           \\
    ss                & Show detailed network info  & ss -s          & Summary of network info             \\
    ifconfig          & Network Interface           & ip             & Enhanced ipconfig/network info.     \\
    \multicolumn{2}{|l|}{nmap -A -T4  scanme.nmap.org } & \multicolumn{2}{l|}{scan host ports - os, version and traceroute}\\[6pt]
   \hline
\end{tabular}
\end{table}

The commands above are a summary list of command line tools that can be used
for networking purposes. In particular, nmap and ncat (note, another 
version of ncat called nc exists, but is not always as reliable), 
will be useful for this assignment. Some of these commands may need to 
be installed using the package management for your machine.



%%% Question 1
\section*{Network Connectivity}
The goal of this exercise is to first introduce you to some useful network
command line tools, to help you explore and debug network programs, and 
then get you to write your first very simple network program. 
\begin{questions}
    \question
     Download the server.cpp file attached to this assignment.
     This is a very simple TCP/IP server, which listens for incoming
     TCP connections on port 5000.  Compile and run it using the commands:
     \begin{lstlisting}
        Linux and Windows/Linux:
           g++ -Wall -std=c++11 server.cpp -o server
        OSX
           g++ -std=c++11 server.cpp -o server

        ./server
     \end{lstlisting}

     Note, you may need to install g++ (On OSX use brew).

     Now connect to it from the same machine you ran the server on
     using the standard loopback network interface "127.0.0.1"
     by performing the following steps.

     From a separate terminal, verify that you can connect to the server 
     which is listening on port 5000, using the ncat command: 

     \begin{lstlisting}
        ncat 127.0.0.1 5000  
     \end{lstlisting}
     
     The server should tell you a client has connected.
     Now on a third terminal, run the tcpdump command (which may require sudo privilege) 
     to monitor the traffic between the client and the server, using the following incantation:

     \begin{lstlisting}
        tcpdump -X -i lo host 127.0.0.10 and port 5000
     \end{lstlisting}

     The server supports one command, SYS $<command>$ which will run the one word command
     specified on the server. Enter a command of your choice (eg. who, ls, w, etc), and 
     observe the results on your terminal. 

 \begin{parts}    
     \part[10] For full marks on this question, repeat the above series of steps,
     connecting to the server from a \textbf{different} machine than the one
     the server is running from, and submit a screen capture
     of each terminal, ncat, ./server and tcpdump (three in total).

     If for \textbf{any} reason you are not able to connect from a different machine,
     submit the screen captures from the same machine as above, and a convincing explanation,
     including the name of the TA you spoke to to get help, of the issue you ran into.
 \end{parts}

\newpage

%%% Question 2
    \question
     For the second part of this assignment, you need to write a client
     program that connects to the server remotely, in the same way you used
     ncat to do. All the client needs to do is to connect to the server, 
     and send a SYS command to the server. For full marks for this question
     submit:

 \begin{parts}
     \part[10] Client code connecting as described above

     \part[2] Screen shot of server receiving command and executing it

     \part[3] tcpdump of command being sent to server

 \end{parts}

%%% Question 3

 \question
      For the third part of this assignment, you need to modify both the server,
      and the client you have just written. 

      The server should be modified to send the client the output from the command
      executed as a response and also to handle parameters on the command being sent
      from the client. For example, "ls -sal". The client should be modifed to receive the output of
      the commend from the server,
      and to print it out on the comand line, and then be ready to accept and send
      another command. That is, the client operates in a loop, send command, receive
      results, print, send command, etc., as network clients often do.

 \begin{parts}
      \part[5] Modified server code meeting above specification
      \part[5] Modified client code meeting above specification
      \part[5] tcpdump of at least two commands being sent to server and response 
      being received.
 \end{parts}
%%% Question 4
 \question
     In addition to the marks for the individual question, the following marks will be
     awarded:
 \begin{parts}
     \part[2] A Makefile is included which compiles the code submitted
     \part[2] README file is provided explaining how to compile and run submitted code. File
     should include command line commands used to compile (IDE's will not be accepted), and
     instructions on how to run the programs.
     \part[2] Code compiles using the Makefile, and runs following instructions in README file
     \part[4] Code is clearly structured (no 200 line functions), and well commented. Variable
     names are informative, and each function/class has a header describing its purpose.
 \end{parts}

\end{questions}
\end{document}
