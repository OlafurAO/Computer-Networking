\documentclass[9pt, addpoints]{exam}
\usepackage[english]{babel}
\usepackage[utf8x]{inputenc}
\usepackage{graphicx,lastpage}
\usepackage{hyperref}
\usepackage{amsmath}
\usepackage{amsthm}
\usepackage{amsfonts}
\usepackage{amssymb}
\usepackage{scrextend}
\usepackage{mathrsfs}
\usepackage{hhline}
\usepackage{booktabs} % book-quality tables
\usepackage{units}    % non-stacked fractions and better unit spacing
\usepackage{multicol} % multiple column layout facilities
\usepackage{lipsum}   % filler text
\usepackage{varwidth} % centering for itemize
\usepackage{listings}
\usepackage[linewidth=1pt]{mdframed}

\renewcommand{\qedsymbol}{$\blacksquare$}

\qformat{\thequestion\dotfill \emph{\totalpoints\ points}}
\pagestyle{headandfoot}
\header{T-409-TSAM}{Assignment 1}{\thepage/\numpages}
\runningheadrule
\firstpagefooter{}{}{}
\runningfooter{}{Page \thepage\ of \numpages}{}

\graphicspath{{../}{Figures/}}
\title{Assignment 1}

\begin{document}
\noindent
\begin{minipage}[l]{.11\textwidth}%
\noindent
    \includegraphics[width=\textwidth]{HR}
\end{minipage}%
%\hfill
\begin{minipage}[r]{.6\textwidth}%
\begin{center}
    {\large\bfseries Department of Computer Science \par
    \large Computer Networks \\[2pt]
    \large Due: Sunday 1st September {23.59}
    }
\end{center}
\end{minipage}%
\fbox{\begin{minipage}[l]{.4\textwidth}%
\noindent
    {\bfseries Your name:}\\[2pt]
TA Name:    \\
Time Taken: \\
{\footnotesize Estimated Time: {10 hours}}
\end{minipage}}%

\large     
\vspace{2cm}
\begin{center}
    \begin{minipage}{40em}
        \begin{center}
            This is an individual assignment and should be submitted as a pdf using Canvas.  
        \end{center}
        
        \vspace{6pt}
    For those who like to dabble in the dark arts, the latex version 
    is also available. You may submit in any legible form you wish.
    
        \vspace{6pt}
    Marks are awarded for question difficulty. While there is 
    typically a relationship between difficulty and length of answer,
    it may not be a strong one. Always justify your answer if necessary,
    especially with somewhat open ended design questions.

    \vspace{6pt}
    Optional: Please include a rough estimate of how long it took you do 
    the assignment so that we can calibrate the work being assigned 
    for the course. (The estimated time is provided purely as a guideline.)
    \par
    \vspace{12pt}
    \end{minipage}
\end{center}

\vspace{4cm}
\begin{center}
    \gradetable[h]
\end{center}
\newpage
\section*{Introduction}


%%% Question 1
\section*{Network Transport Times}
\begin{questions}
    \question
\begin{parts}
     \part[2]
     {traceroute and ping are command line tools to show the
              network path to another host.  
              
    Perform a traceroute to each of the following hosts. For
    each host, give at least one intermediate countries the network
    packets are going through.

    \begin{enumerate}{}
        \item mel1.speedtest.telstra.net
        \item 130.69.0.0
    \end{enumerate}
    }

    \part[2]{What does a "* * *" line in the traceroute response mean?}

    \part[2] 
    {
     Using the ping command, what is the round trip time (RTT) to the
     following hosts?
     \begin{enumerate}{}
            \item mel1.speedtest.telstra.net
            \item per1.speedtest.telstra.net
     \end{enumerate}
     }
     \part[2]{Both the hosts are in Australia, one is in Melbourne, the
     other in Perth. If the speed of light in a vacuum is ~300,000,000 m/s 
     and the core index of refraction of fiber-optic cable in the Australian
     backbone is 1.50, approximately how far is Perth from Melbourne?}

     \part[4]{CA Technologies provides a site which allows you to ping a
      publicly accessible Internet host from different hosts worldwide in
      order to measure local response time for your users. 

    \url{https://asm.ca.com/en/ping.php}

    Use this site to ping:
    \begin{itemize}{}
        \item www.ru.is
        \item www.mit.edu
    \end{itemize}

    By examining the difference in ping response times for www.ru.is, 
    in what country is this host actually located?


    In what countries does MIT appear to be located?


    What service is MIT using to do this?
    }


\end{parts}

%%% Question 2

     \section*{Network Throughput}
  \question
  \begin{parts}
      \part[2]
      You need to transfer a geophysical
      dataset of 100TB stored on disk in Iceland to the Norwegian Metrology
      Office.  How long will it take to transfer this dataset to Norway
      assuming a 1Gbps connection, and 15\% protocol overhead?
      \vspace{8pt}
      \par
     \part[2]
      Ref: \url{https://en.wikipedia.org/wiki/Linear_Tape-Open}
      \vspace{8pt}

      \vspace{\stretch{4}}

      An industry standard tape (circa 2018) can hold 12TB of
      data on a single cartridge.

      Assuming a best case scenario of 3 hours ground transport time to
      Keflavik airport and 3 hours from Oslo to destination company, with 
      a scheduled flight time also of 3 hours. 
      How much data do you need before it is quicker to send the data by 
      tape than transfer it over the network? (Ignore time to read and write 
      the tape.)

    \vspace{\stretch{4}}

     \part[2] Tannenbaum in Computer Networks wisely advises never to overlook
      the speed of sending data by existing transport networks - planes
      in this case. However, his example overlooks the time taken to
      create the tapes in the first place.

      Assuming that the maximum writing and reading speed for a tape is
      900(MB/s), how long does it actually take to transfer the data
      to Norway including the time to read and write the tapes?
    \vspace{\stretch{4}}

      \part[1] What is the new break even amount for sending data by planes?
    \vspace{\stretch{4}}

  \end{parts}

%%% Question 3

  \section*{Network Engineering}
 \question
   An ISP is statistically multiplexing its customers over 10Gb links.
   You have been asked to calculate how many customers it can afford to assign 
   to each link, and maintain a reasonable level of service to each one.

   Nominally, each customer is being sold a 1Gb link.

  \begin{parts}
  \part[1]{If each customer is to be guaranteed access to 1Gb at any time,
      how many customers can the ISP provision per 10Gb link.}

  \part[2]{Assume that the 10Gb link costs the ISP 200,000 ISK/month,
      and the ISP needs to make 20\% profit to cover all overheads.
      What is the smallest number of customers that the ISP provision 
      for each 10Gb link, and still meets its profit targets, if the 
      ISP charges each customer 7,000ISK for their Internet service?
  }
  \part[2]{What is the maximum speed each customer will be able to
      download data at, assuming all customers are maximising their
      network connection?}


  \part[4] The ISP decides that on average each customer will use
      their link 10\% of the time, evenly distributed over the day.
      Assuming this is correct, how many customers can the ISP now 
      provision and still maintain the illusion that they have access
      to 1Gb each? 

    \part[2]If the ISP has a mixture of business and household customers,
      how should it assign the different types of customer to its links
      to improve performance?

  \end{parts}

\end{questions}
%%% Question 4
 \section*{Bonus Question: 1 Bonus Mark}

 Referring back to question 1, and the time taken for a packet to travel
 to and across Australia.  Using traceroute create a table of round trip 
 times to the two hosts in question 1, and to and from one of the London 
 switches performed at least four different times of the day (it does not
 have to be the same day), separated by at least 2 hours
 from each other. That is you are capturing the time from Iceland to London,
 and from London to Australia for the same route.

 \vspace{18pt}
 Include your measurements in a table. With reference to these measurements:
 \vspace{18pt}

 What is the fastest and slowest time of day to send traffic to Australia?
 
 \vspace{18pt}

 Where is the congestion occurring that slows down the traffic?

\end{document}
